\documentclass[draft,a4paper]{article}
\usepackage[utf8]{inputenc}
\usepackage{amsmath}
\usepackage{graphicx}
\usepackage{url}
\usepackage{fancyhdr}
\usepackage{datetime}

\newcommand{\email}[1]{#1}
\newcommand{\course}{Operating Systems, 5dv171}
\newcommand{\semester}{Fall 2017}
\newcommand{\assignment}{Project Linux Kernel Module - part 1}
\newcommand{\auth}[2]{#1, \email{#2}}
\newcommand{\authors}{
  \auth{Igor Ramazanov}{ens17irv@cs.umu.se}\\
  \auth{Erik Ramos}{c03ers@cs.umu.se}\\
  \auth{Bastien Harendarczyk}{ens17bhk@cs.umu.se}\\
  \vspace{1em}
}
\newcommand{\lecturer}{Jan-Erik Moström}
\newcommand{\assistants}{Adam Dahlgren}
\newcommand{\codebase}{https://git.cs.umu.se/c03ers/5dv171-project}

\setlength{\parindent}{0em}
\setlength{\parskip}{1em}

%\newcommand{\frontpage}{
\title{
  \vspace{4em}
  \pagenumbering{gobble}
  \begin{center}
  {\LARGE \bf \course{}, \semester}\\
  {\Large \bf \assignment}\\
  \vspace{6em}
  {\normalsize
  {\bf Teacher}\\
  {\lecturer}\vspace{1em}\\
  {\bf Teaching assistent}\\
  {\assistants}\vspace{1em}\\
  {\bf Group members}\\
  \authors
  {\bf Gitlab repository}\vspace{-1em}\\
  {\codebase}}
  \end{center}
  \pagebreak
  \pagenumbering{arabic}
}

\author{}
\date{\today}
\pagestyle{fancy}
\renewcommand{\headrulewidth}{0em}
\fancyhead[R]{\the\year-\twodigit{\the\month}-\twodigit{\the\day}}

\begin{document}
\maketitle

\section*{User's guide}
The project consists of two parts - the kernel module and the user space
test program. All files required to build both parts can be found in the gitlab
repository.

To build the project, begin by cloning the repositorybyt typing the following in
a terminal:
\begin{center}
{\tt git clone \codebase}
\end{center}

\section*{Problem description}
The goal with this project is to write a Linux Kernel Module (LKM) that allows
different processes running on the same linux machine, to access shared data
through key-value mappings. The system must be robust enough, that multiple
processes, running on multiple processors, will not accidentally corrupt the
stored data.

This first part of the project report describes how the LKM represents these
mappings in kernel space, as well as how the user space applications can
access them.

\section*{Solution}
The following three methods of user to kernel space communication were
considered for this project:
\begin{enumerate}
  \item Device files
  \item Proc files
  \item Netlink sockets
\end{enumerate}
Using one of the filesystems to solve the problem was appealing, primarly
because much of the heavy lifting would be done by system calls aleady
available in Linux. However, since device files usually repressent an actual,
physical device and proc files are meant to contain information about processes,
none of these seemed like a really good fit.

Because the kernel module would provide user space applications with a service,
in much the same way that an Internet server would to its clients, Netlink
sockets seemed like a much better fit. They would however, be much more
complicated to set up and use.

\section*{System limitations}

\section*{Discussion}

\end{document}
